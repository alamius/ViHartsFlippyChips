    \begin{figure}
        \begin{tikzpicture}[scale=5]
            \draw
                let \p{P} = (0, 0),
                    \p{p} = (1, 2),
                    \p{Q} = (1, 0),
                    \p{q} = (-2, -1)
                in
                    node[inner sep=0] (P)  at (\p{P}) {$\bullet$}
                    node[inner sep=0] (p)  at ({\x{P}+\x{p}/3}, {\y{P}+\y{p}/3}) {}
                    node[inner sep=0] (Q)  at ({\x{Q}},     {\y{Q}}) {$\bullet$}
                    node[inner sep=0] (q)  at ({\x{Q}+\x{q}/3}, {\y{Q}+\y{q}/3}) {};
            \draw[-latex] (P) node[below left]  {$P = 0$} -- (p) node[midway, left] {$p = 1+2i$};
            \draw[-latex] (Q) node[above left] {$Q = 1$} -- (q) node[midway, below right] {$q = -2-i$};
            \draw[domain=0:1, samples=\samplesTikz, variable=\t]
                let \p{P} = (0, 0),
                    \p{p} = (1, 2),
                    \p{Q} = (1, 0),
                    \p{q} = (-2, -1)
                in plot ({
                    \x{P} + \x{p}*(\t) + (3*\x{Q} - 3*\x{P} - 2*\x{p} - \x{q})*(\t)*(\t) + (2*\x{P} - 2*\x{Q} + \x{p} + \x{q})*(\t)*(\t)*(\t)
                }, {
                    \y{P} + \y{p}*(\t) + (3*\y{Q} - 3*\y{P} - 2*\y{p} - \y{q})*(\t)*(\t) + (2*\y{P} - 2*\y{Q} + \y{p} + \y{q})*(\t)*(\t)*(\t)
                });
        \end{tikzpicture}
        \caption{Spline $\Lspline_{0,1,1+2i,-2-i}$}
        \label{fig:spline1}
    \end{figure}
    The Line that is the basis for a Chip is made from a Series of Splines. Splines are functions that take a number from 0 to 1 and return a smoothly changing 2D-Value from a Startpoint $P$ to a Endpoint $Q$. They start at $P$ going out with a speed and direction defined by $p$ and enter $Q$ following $q$. I'll either write them as $\Lspline_{PQpq}(t)$ or simply as $\widetilde{PQ}(t)$ if $p$ and $q$ are inferable from context.
