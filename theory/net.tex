\section{Net}
	\begin{figure}
		\begin{tikzpicture}[scale=9]
			\draw[-latex]
			let
				\p{P} = (0, 0),
				\p{p1} = (2, -.5),
				\p{p2} = (-.5, -1.5),
				\p{Q} = (1, -.2),
				\p{q1} = (0, 1.5),
				\p{q2} = (2, 0),
				\p{R} = (.5, 1),
				\p{r1} = (-1, -1),
				\p{r2} = (-1, 2)
			in
				node[inner sep=0] (P)  at (\p{P}) {$\bullet$}
				node[inner sep=0] (p1) at ({\x{P}+\x{p1}/3}, {\y{P}+\y{p1}/3}) {}
				node[inner sep=0] (p2) at ({\x{P}+\x{p2}/3}, {\y{P}+\y{p2}/3}) {}
				node[inner sep=0] (Q)  at ({\x{Q}},       {\y{Q}}) {$\bullet$}
				node[inner sep=0] (q1) at ({\x{Q}+\x{q1}/3}, {\y{Q}+\y{q1}/3}) {}
				node[inner sep=0] (q2) at ({\x{Q}+\x{q2}/3}, {\y{Q}+\y{q2}/3}) {}
				node[inner sep=0] (R)  at (\p{R}) {$\bullet$}
				node[inner sep=0] (r1) at ({\x{R}+\x{r1}/3}, {\y{R}+\y{r1}/3}) {}
				node[inner sep=0] (r2) at ({\x{R}+\x{r2}/3}, {\y{R}+\y{r2}/3}) {};
			\draw[-latex] (P) node[below left] {$P$}  -> (p1) node[right]      {$\vec p_1$};
			\draw[-latex] (P)                         -> (p2) node[midway, left] {$\vec p_2$};
			\draw[-latex] (Q) node[above right] {$Q$} -> (q1) node[midway, right]      {$\vec q_1$};
			\draw[-latex] (Q)                         -> (q2) node[midway, above]      {$\vec q_2$};
			\draw[-latex] (R) node[right] {$R$}       -> (r1) node[left] {$\vec r_1$};
			\draw[-latex] (R)                         -> (r2) node[midway, below left] {$\vec r_2$};
			\draw[
				domain=0:1.03,
				% domain=-.1:1.1,
				samples=\samplesTikz,
				variable=\t
			]
			let
				\p{P} = (0, 0),
				\p{p1} = (2, -.5),
				\p{p2} = (-.5, -1.5),
				\p{Q} = (1, -.2),
				\p{q1} = (0, 1.5),
				\p{q2} = (2, 0),
				\p{R} = (.5, 1),
				\p{r1} = (-1, -1),
				\p{r2} = (-1, 2)
			in
				\Spline{\x{P}}{\y{P}}{\x{Q}}{\y{Q}}{\x{p1}}{\y{p1}}{\x{q2}}{\y{q2}}
				\Spline{\x{Q}}{\y{Q}}{\x{R}}{\y{R}}{\x{q1}}{\y{q1}}{\x{r2}}{\y{r2}}
				\Spline{\x{R}}{\y{R}}{\x{P}}{\y{P}}{\x{r1}}{\y{r1}}{\x{p2}}{\y{p2}}
				% generated by test_PQR from main.cpp with F[v_].latex()
				%F
				\Spline{0}{0}{0.96875}{0.104687}{1.625}{0}{1.75}{0.25} node[above right] {$F(t, \frac 14)$}
				\Spline{0}{0}{0.875}{0.3375}{1.25}{0.5}{1.5}{0.5}      node[above right] {$F(t, \frac 12)$}
				\Spline{0}{0}{0.71875}{0.601562}{0.875}{1}{1.25}{0.75} node[above right] {$F(t, \frac 34)$}
				\Spline{0}{0}{0.5}{1}{0.5}{1.5}{1}{1}                  node[above right] {$F(t, 1)$}
				%E
				\Spline{0.34375}{-0.101562}{0.101562}{0.320312}{0.09375}{0.375}{-0.4375}{0.21875} node[above left] {$E(\frac 14, v)$}
				\Spline{0.5}{-0.1625}{0.1875}{0.5625}{0.125}{0.75}{-0.75}{0.625}                  node[above left] {$E(\frac 12, v)$}
				\Spline{0.65625}{-0.192188}{0.304688}{0.773438}{0.09375}{1.125}{-0.9375}{1.21875} node[above left] {$E(\frac 34, v)$}
				\Spline{1}{-0.2}{0.5}{1}{0}{1.5}{-1}{2}                                           node[left] {$E(1, v)$}
				% \foreach \t in {.1, .2, .3, .4}
				%     \Spline{\x{P}}{\y{P}}{\x{Q}}{\y{Q}}{\x{p1}}{\y{p1}}{\x{q2}}{\y{q2}}
			;
		\end{tikzpicture}
		\caption{Net of $\Espline_{PQRp_1p_2q_1q_2r_1r_2}(t, v)$ and $\Fspline_{PQRp_1p_2q_1q_2r_1r_2}(t, v)$}
		\label{fig:eta}
	\end{figure}
	To finally color a Face, I define a Net of Splines going either out from every Corner or in waves around every Corner. In the Example below, the Corner is $P$ with $Q$ on one and $R$ on the other side. So, one line is going out from $P$ to $Q$ along the Spline that connects them, and one from $P$ to $R$:
	$$\begin{aligned}
		\Fspline(t, 0) =& \Lspline_{P,Q,p_1,q_2}(t) \\
		\Fspline(t, 1) =& \Lspline_{P,R,-p_2,-r_1}(t) \\
	\end{aligned}$$
	Then, the rest are Interpolated between those two: The Begin stays at $P$, the First Control ($p$) changes from $p_1$ (leads towards $Q$) to $-p_2$ (to $R$) with a Parameter $v$: I call that direction $c(v)$. The End Point moves from $Q$ to $R$ along the Spline defined by the other directions of $Q$ and $R$, those that don't lead towards $P$: $q_1, r_2$: That's $C(v)$. And the direction of the Spline at the End Point is Interpolated between the direction of $\widetilde{PQ}$ at the End and $\widetilde{PR}$ at the End: $d(v)$. It is important to use Interpolated directions and not just Right Angles here, because near the actual Corners, they need to match the Edges' Angles, which don't got out neccessarily perpendicularily.
	$$\begin{aligned}
		C(v) :=& \Lspline_{QRq_1r_2}(v) \\
		c(v) :=& -p_2v + p_1(1-v) \\
		d(v) :=& -r_1v + q_2(1-v) \\
		\Fspline(t, v) =& \Lspline_{P,C(v),c(v),d(v)}(t) \\
	\end{aligned}$$
	$\Espline(t, v)$ is defined similarly, but the Splines go from $\widetilde{PQ}$ to $\widetilde{PR}$ and not along these, they move out in waves and not in beams. It's Implementation is currently faulty and I have not used it because F was sufficient for me, but it could be worth looking into.
	$$\begin{aligned}
		A(t) :=& \Lspline_{PQp_1q_2}(t) \\
		B(t) :=& \Lspline_{PR-p_2-r_1}(t) \\
		a(t) :=&  q_1t - p_2(1-t) \\
		b(t) :=&  r_2t - p_1(1-t) \\
		\Espline(t, v) =& \Lspline_{A(t),B(t),a(t),b(t)}(v) \\
	\end{aligned}$$
	When there is a structure to every Corner of a Face, the only thing missing is the correct Color assignment for the little Quadritalerals...
