\section{Graph}
    \begin{figure}
        \begin{tikzpicture}[scale=8]
            \node (A) at (0.4, 0.8) {$\bullet$};
            \node (B) at (0.1, 1) {$\bullet$};
            \node (C) at (0.4, 0.2) {$\bullet$};
            \node (D) at (0.7, 1) {$\bullet$};
            \node (E) at (0.7, 0.2) {$\bullet$};
            \draw[-latex] (A) node[below] {$A$} .. controls +({-1*\controlfactor}, {0*\controlfactor}) and +({-0.5*\controlfactor}, {-0.5*\controlfactor}) .. (B);
            \draw[-latex] (B) node[below] {$B$} .. controls +({0.5*\controlfactor}, {0.5*\controlfactor}) and +({-1*\controlfactor}, {-0*\controlfactor}) .. (C);
            \draw[-latex] (C) node[below] {$C$} .. controls +({1*\controlfactor}, {0*\controlfactor}) and +({1*\controlfactor}, {-0*\controlfactor}) .. (D);
            \draw[-latex] (D) node[below] {$D$} .. controls +({-1*\controlfactor}, {0*\controlfactor}) and +({-1*\controlfactor}, {-0*\controlfactor}) .. (E);
            \draw[-latex] (E) node[below] {$E$} .. controls +({1*\controlfactor}, {0*\controlfactor}) and +({1*\controlfactor}, {-0*\controlfactor}) .. (A);
            \node (P) at (0.1846, 0.808273) {$\bullet$};
            \node (Q) at (0.560517, 0.258828) {$\bullet$};
            \node (R) at (0.776839, 0.544064) {$\bullet$};
            \node (S) at (0.477184, 0.790679) {$\bullet$};
            \draw (P) node[above right] {$P$};
            \draw (Q) node[above] {$Q$};
            \draw (R) node[right] {$R$};
            \draw (S) node[above right] {$S$};
            \node (PQS) at (.32, .5) {$(PQS)$};
            \node (QRS) at (.6,  .5) {$(QRS)$};
            \node (QR)  at (.72, .3) {$(QR)$};
            \node (RS)  at (.7,  .85){$(RS)$};
            \node (P)   at (.12, .9) {$(P)$};
            \node (PPSRQ)at(.1,  .2) {$(PPSRQ)$};
        \end{tikzpicture}
        \caption{}
        \label{}
    \end{figure}
    When all the Intersections $P, Q, R, ...$ in the Chip are found, that gives away neither the Edges nor the Faces immediately. For every Intersection, the parameters $t$ and $u$ are given and also , more importantly, the Points $A$ and $C$, they come from. with that information, an algorithm can follow the Line from Spline $\widetilde{AB}(t)$ on until it finds andother intersection $(A_2, C_2, t_2, u_2)$. In the Example, $P$ is roughly $\widetilde{AB}(0.26) = \widetilde{BC}(0.39)$.
